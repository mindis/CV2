\section{ICP}
\label{ICP}
As an extension to the methods that were discussed in the lectures for the 3D reconstruction we have chosen to implement the Iterative Closest Point algorithm using the implentation of \fixme{"Implementation of a 3D ICP-based Scan Matcher" by Zhongjie and Shou-Yu.}. It is a algorithm that will minimize the distance between two point clouds. We hoped that this would work on the structure point clouds generated by the structure from motion. It works as follows:

\begin{itemize}
	\item For $N$ iterations, or until converged do the following:
	\begin{itemize}
		\item For every point in the source cloud find the closest point in the reference cloud.
		\item Estimate the combination of rotation and translation using a mean squared error cost function that will best align each source point to its match found in the previous step.
		\item Transform the source points using the obtained transformation.
	\end{itemize}
\end{itemize}

If this would all go well we could use it to reconstruct the teddybear from the 3D point clouds.
